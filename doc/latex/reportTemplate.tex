\documentclass[11pt]{article}
\usepackage{geometry}                
\geometry{letterpaper}                   

\usepackage[english]{babel}	
\usepackage{graphicx}
\usepackage{german}
\usepackage[utf8]{inputenc} 
\usepackage{fancyhdr}
\usepackage{tabularx}
\usepackage{amssymb}
\usepackage{epstopdf}
\usepackage{natbib}
\usepackage{amssymb, amsmath}

\DeclareGraphicsRule{.tif}{png}{.png}{`convert #1 `dirname #1`/`basename #1 .tif`.png}

%\title{Title}
%\author{Name 1, Name 2}
%\date{date} 

\begin{document}



\thispagestyle{empty}

\begin{center}
\includegraphics[width=5cm]{ETHlogo.eps}

\bigskip


\bigskip


\bigskip


\LARGE{ 	Lecture with Computer Exercises:\\ }
\LARGE{ Modelling and Simulating Social Systems with MATLAB\\}

\bigskip

\bigskip

\small{Project Report}\\

\bigskip

\bigskip

\bigskip

\bigskip


\begin{tabular}{|c|}
\hline
\\
\textbf{\LARGE{Swiss Rail Network Formation with}}\\
\textbf{\LARGE{Physarum Polycephalum}}\\
\\
\hline
\end{tabular}
\bigskip

\bigskip

\bigskip

\LARGE{Lucas Böttcher \\ Simon Roth \\ Gabriela Schär}



\bigskip

\bigskip

\bigskip

\bigskip

\bigskip

\bigskip

\bigskip

\bigskip

Zurich\\
December 2012\\

\end{center}



\newpage

%%%%%%%%%%%%%%%%%%%%%%%%%%%%%%%%%%%%%%%%%%%%%%%%%

\newpage
\section*{Agreement for free-download}
\bigskip


\bigskip


\large We hereby agree to make our source code for this project freely available for download from the web pages of the SOMS chair. Furthermore, we assure that all source code is written by ourselves and is not violating any copyright restrictions.

\begin{center}

\bigskip


\bigskip


\begin{tabular}{@{}p{6cm}@{}p{6cm}@{}@{}p{6cm}@{}}
\begin{minipage}{6cm}
 \large Lucas Böttcher

\end{minipage}
& 
\begin{minipage}{6cm}
\large Simon Roth

\end{minipage}
&
\begin{minipage}{6cm}
\large Gabriela Schär

\end{minipage}
\end{tabular}
\end{center}
\newpage

%%%%%%%%%%%%%%%%%%%%%%%%%%%%%%%%%%%%%%%



% IMPORTANT
% you MUST include the ETH declaration of originality here; it is available for download on the course website or at http://www.ethz.ch/faculty/exams/plagiarism/index_EN; it can be printed as pdf and should be filled out in handwriting


%%%%%%%%%% Table of content %%%%%%%%%%%%%%%%%

\tableofcontents

\newpage

%%%%%%%%%%%%%%%%%%%%%%%%%%%%%%%%%%%%%%%



\section{Abstract}

Generating a biologically inspired model of the Swiss railroad network depending on population growth. By using a biological (based on Physarum polycephalum) model, we want to simulate future scenarios of the Swiss railroad network. Probably a problem in future would be, that in fact of the population growth more people will use the public transport system. So we think it is necessary to simulate the main transport lines, to see where the system should be improved.




\section{Individual contributions}

\section{Introduction and Motivations}

\section{Description of the Model}

The mathematical model is based on \cite{network_model}. The urban areas (food sources) are the independent variables. By using theorems from hydro dynamics there exists three types of dependent variables for each node: conductivity, length and pressure.

As mentioned in \cite{network_model}, the flux prefers junctions with high efficiency (e.g. short connections). Because the flux in the system is constant, the junctions with high efficiency grow (getting thicker and more flux goes through), while the junctions with low efficiency shrink and disappear.


\section{Implementation}

\section{Simulation Results and Discussion}

\section{Summary and Outlook}

\section{References}

\bibliographystyle{plain}
\bibliography{matlabbib}






\end{document}  



 
