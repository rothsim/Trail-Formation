\documentclass[11pt]{scrartcl}
\usepackage{geometry}                
\geometry{letterpaper}                   

\usepackage[english]{babel}	
\usepackage{graphicx}
\usepackage{german}
\usepackage[utf8]{inputenc} 
\usepackage{fancyhdr}
\usepackage{tabularx}
\usepackage{amssymb}
\usepackage{epstopdf}
\usepackage{natbib}
\usepackage{amssymb, amsmath}
\usepackage{float}
\usepackage{amsmath}
\usepackage{booktabs}

\DeclareGraphicsRule{.tif}{png}{.png}{`convert #1 `dirname #1`/`basename #1 .tif`.png}

%\title{Title}
%\author{Name 1, Name 2}
%\date{date} 

\renewcommand*{\capfont}{\normalfont}
\renewcommand*{\caplabelfont}{\sffamily\bfseries}


\begin{document}



\thispagestyle{empty}

\begin{center}
\includegraphics[width=5cm]{ETHlogo.eps}

\bigskip


\bigskip


\bigskip


\LARGE{ 	Lecture with Computer Exercises:\\ }
\LARGE{ Modelling and Simulating Social Systems with MATLAB\\}

\bigskip

\bigskip

\small{Project Report}\\

\bigskip

\bigskip

\bigskip

\bigskip


\begin{tabular}{|c|}
\hline
\\
\textbf{\LARGE{Swiss Rail Network Formation with}}\\
\textbf{\LARGE{Physarum Polycephalum}}\\
\\
\hline
\end{tabular}
\bigskip

\bigskip

\bigskip

\LARGE{Lucas Böttcher \\ Simon Roth \\ Gabriela Schär}



\bigskip

\bigskip

\bigskip

\bigskip

\bigskip

\bigskip

\bigskip

\bigskip

Zurich\\
December 2012\\

\end{center}



\newpage

%%%%%%%%%%%%%%%%%%%%%%%%%%%%%%%%%%%%%%%%%%%%%%%%%

\newpage
\section*{Agreement for free-download}
\bigskip


\bigskip


\large We hereby agree to make our source code for this project freely available for download from the web pages of the SOMS chair. Furthermore, we assure that all source code is written by ourselves and is not violating any copyright restrictions.

\begin{center}

\bigskip


\bigskip


\begin{tabular}{@{}p{6cm}@{}p{6cm}@{}@{}p{6cm}@{}}
\begin{minipage}{6cm}
 \large Lucas Böttcher

\end{minipage}
& 
\begin{minipage}{6cm}
\large Simon Roth

\end{minipage}
&
\begin{minipage}{6cm}
\large Gabriela Schär

\end{minipage}
\end{tabular}
\end{center}
\newpage

%%%%%%%%%%%%%%%%%%%%%%%%%%%%%%%%%%%%%%%



% IMPORTANT
% you MUST include the ETH declaration of originality here; it is available for download on the course website or at http://www.ethz.ch/faculty/exams/plagiarism/index_EN; it can be printed as pdf and should be filled out in handwriting


%%%%%%%%%% Table of content %%%%%%%%%%%%%%%%%

\tableofcontents

\newpage

%%%%%%%%%%%%%%%%%%%%%%%%%%%%%%%%%%%%%%%



\section{Abstract}

Like most countries in the world, Switzerland is growing. Cities are getting more inhabitants, are growing together or are arising new. This growth in population causes an increasing use of the public transport system, which has to be adapted to the new conditions. The main goal of this project is the simulation of the Swiss rail network depending on population growth. The network is simulated with a biological inspired model based on Physarum polycephalum. This slime mold is a large single-celled amoeboid organism that forages for food sources. To maximize the searched area, it explores it's environment with a relatively continous foraging margin. It's forming different junctions and nodes to reduce the overall length of the connecting network \cite{network_tokyo}. This principle is adapted to the main railroads in Switzerland. It can be shown...

 


\section{Individual contributions}



\section{Introduction and Motivations}
Everyday thousands of commuters use the Swiss rail network to reach their workspace or the university, mainly the same railroad is used. But is the actual situation of the network the most efficient or have different connection between cities a higher performance?

We assume, that the rail network is developed in a way that the connections between cities are most efficient. Certain biological organism connect their food sources in a similar way. In this project we simulate a biological organism creating its own efficient connections between the cities. This approach leads to the following main questions:

\begin{enumerate}
	\item Is the biological model a good approximation to simulate the network compared to reality?
	\item Are there any new built or destroyed connections in the network because of future population growth?
\end{enumerate}

To answer these questions, the results of the simulations are compared with the Swiss rail network. For this comparison geodata from geodata.ethz.ch~\cite{gis_data} are used. This data are processed with \textit{Arcgis 10.1} to get the basis for the simulation.

Cities are choosen based on the numbers of inhabitants (more than 10000) in 2011 and the presence of a railaway station. Additional to these some cities with less than 10000 inhabitants are added because of actual importance for the Swiss rail network\,\cite{bfs}. ~\\
~\\
To answer the second question, ....~\\
~\\
We expect that the most efficient network created by the simulation is a good approximation of the actual Swiss railroad network. We also expect that the network won't change even when the population is growing because a network changes the most at the beginning of developing. And the Swiss railroad network reaches the end of developement already.




\section{Description of the Model}

The model is inspired of the physarum polycephalum and the way it's searching for food. The organism have been subjected to successive rounds of evolutionary selection and have found an appropriate balance between efficiency, cost and resilience. The plasmodium contains a network of tubes, which enables chemical signals and nutrients to circulate through the organism. If some of the tubes have found food sources, this implies a positive feedback to the system. This tubes are getting thicker so the flux of nutrients increase. Other tubes which don't connect to a food sources are shrinking and tend to disapear because there is no flux available. Experiments showed two empirical rules. 1) Tubes with no connection to a source disappear and 2) if there are two tubes connecting the same source, the longer one disappears. This led to useful approaches to problem-solving like optimization of railroad systems or networks in general.~\\
~\\
The mathematical model is based on \textit{Physarum solver: A biologically inspired method of road-network navigation} \cite{network_model}. Figure \ref{fig:schema} shows the concept of the mathematical model. There are different nodes. The first two nodes corresponding to the food sources ($N_1$ and $N_2$) and other nodes ($N_3$, $N_4$ ... $N_j$). The junction between the node $N_i$ and $N_j$ is denoted as $M_{ij}$.

\begin{figure}[H]
	\centering
	\includegraphics[width=0.4\textwidth]{schema.jpg}
	\caption{Concept of the mathematical model. Squares represent ordinary nodes, circles 			represent food sources, each line represent a junction.}
	\label{fig:schema}
\end{figure}

The variable $Q_{ij}$ stands for the flux through the junction $M_{ij}$ from $N_i$ to $N_j$. The flux $Q_{ij}$ is given by an approximately Poiseuille flow where $p_i$ is a pressure at the node $N_i$, $L_{ij}$ is the length and $D_{ij}$ is the conductivity of the junction $M_{ij}$ :

\begin{equation}
	\label{eq:1}
	Q_{ij}=\frac{D_{ij}}{L_{ij}}\left(p_i-p_j\right)
\end{equation}

By considering Kirchhoff's law at each node, there is:

\begin{equation}
	\label{eq:2}
	\sum_{i} Q_{ij}=0 \,\,\,\, \mathrm{if} \left(j\ne 1,2\right)
\end{equation}

$N_1$ is assumed as a source node and $N_2$ as a sink and $I_0$ represent the flux from the source node and is in this model constant. It follows:

\begin{equation}
	\label{eq:3}
	\sum_{i} Q_{i1}+I_0=0, \,\,\,\, \sum_{i} Q_{i2}-I_0=0
\end{equation}

To describe the thickness of the junctions we assumed that the conductivity $D_{ij}$ changes in time according to the flux $Q_{ij}$:

\begin{equation}
	\label{eq:4}
	\frac{dD_{ij}}{dt}=f\left(\mid Q_{ij} \mid \right)-D_{ij}
\end{equation}

where $f\left(\mid Q \mid \right)$ is a increasing function with $f(0)=0$. Here $f\left(\mid Q \mid \right)$ is given by:

\begin{equation}
	\label{eq:5}
	f\left(\mid Q \mid \right)=\frac{\mid Q \mid^\gamma }{1+\mid Q \mid^\gamma}
\end{equation}

The network Poisson equation for the pressure is derived from the equations (\ref{eq:1}), (\ref{eq:2}) and (\ref{eq:3}) as followed:

\begin{equation}
	\label{eq:6}
	\sum_{i} \frac{D_{ij}}{L_{ij}}\left(p_i-p_j\right)= \begin{cases}
										-I_0 & \mathrm{for}\,\, j=1,\\
										I_0 & \mathrm{for} \,\,j=2,\\
										0 & \mathrm{otherwise}
										\end{cases}
\end{equation}

All $p_i$'s can be determined by solving the equation system (\ref{eq:6}) wenn setting $p_2=0$ as a basic pressure level. With this also each $Q_{ij}$ is obtained. They're definend bay the $D_{ij}$'s and $L_{ij}$'s  at each time step. Conductivity is closely related to the thickness of the junctions and so when a conductivity of a junction is zero, it disappears.

\section{Implementation}

To get the basis matrix for the simulation, Swiss geodata\,\cite{gis_data} are processed with \textit{Arcgis 10.1}. Therefor the rasterdata are distinguished in different classes as in Table\, \ref{tab:class} are described.

\begin{table}[H]
	\centering
	\caption{classes of data}
		\begin{tabular}{lll}
		\toprule
		Indentification & Class & Description \\
		\midrule
		0 & Ghostland & Cells in which the slime mold isn't allowed to be\\
		& 		& (foreign coutries, lakes)\\
		1 & Freeland & Cells in which the slime mold is allowed to grow\\
		2 & Junction & Cells in which the slime mold is located, \\
		& & processed in MATLAB\\
		3 & City & Cells which represents food sources\\
		\bottomrule
	\end{tabular}
\label{tab:class}
\end{table}

All the geodata are in vector format. So all lakes and the foreign countries can be erased. It would be possible to erase more types of covering (eg. rivers, rocks) where a train can't ride. The slope is in this case also neglected. So the mountains aren't considerated in the simulation. Then all the choosen cities are imported to \textit{Arcgis}. Therefor the coordinates (Y,X)\,\cite{coordinates} for each city are searched and a buffer of 2500m is layed on them. In this case no city will disapear when the vectordata gets converted to rasterdata with a cellsize of 2500m. So after converting in rasterdata with the classes above, the basis matrix is exported as ASCII-File which can now used in MATLAB as simulation surface.



\begin{figure}[H]
	\centering
	\includegraphics[width=0.7\textwidth]{map_2500_cities.jpg}
	\caption{Rastardata from Arcgis with a gridsize of 2500m - white: Ghostland, green: Freeland, brown: City}
	\label{fig:map_cities}
\end{figure}

\section{Simulation Results and Discussion}

\section{Summary and Outlook}

\appendix
\section{Matlab-Code}

Matlab-Code

\section{Declaration of Originality}



\bibliographystyle{plain}
\bibliography{matlabbib}






\end{document}  



 
