\documentclass[11pt]{scrartcl}
\usepackage{geometry}                
\geometry{letterpaper}                   

\usepackage[english]{babel}	
\usepackage{graphicx}
\usepackage{german}
\usepackage[utf8]{inputenc} 
\usepackage{fancyhdr}
\usepackage{tabularx}
\usepackage{amssymb}
\usepackage{epstopdf}
\usepackage{natbib}
\usepackage{amssymb, amsmath}
\usepackage{float}
\usepackage{amsmath}
\usepackage{booktabs}
\usepackage{pdfpages}
\usepackage{moreverb}
\usepackage{mcode}


\DeclareGraphicsRule{.tif}{png}{.png}{`convert #1 `dirname #1`/`basename #1 .tif`.png}

%\title{Title}
%\author{Name 1, Name 2}
%\date{date} 

\renewcommand*{\capfont}{\normalfont}
\renewcommand*{\caplabelfont}{\sffamily\bfseries}


\begin{document}



\thispagestyle{empty}

\begin{center}
\includegraphics[width=5cm]{ETHlogo.eps}

\bigskip


\bigskip


\bigskip


\LARGE{ 	Lecture with Computer Exercises:\\ }
\LARGE{ Modelling and Simulating Social Systems with MATLAB\\}

\bigskip

\bigskip

\small{Project Report}\\

\bigskip

\bigskip

\bigskip

\bigskip


\begin{tabular}{|c|}
\hline
\\
\textbf{\LARGE{Swiss Rail Network Formation with}}\\
\textbf{\LARGE{Physarum Polycephalum}}\\
\\
\hline
\end{tabular}
\bigskip

\bigskip

\bigskip

\LARGE{Lucas Böttcher \\ Simon Roth \\ Gabriela Schär}



\bigskip

\bigskip

\bigskip

\bigskip

\bigskip

\bigskip

\bigskip

\bigskip

Zurich\\
December 2012\\

\end{center}



\newpage

%%%%%%%%%%%%%%%%%%%%%%%%%%%%%%%%%%%%%%%%%%%%%%%%%

\newpage
\section*{Agreement for free-download}
\bigskip


\bigskip


\large We hereby agree to make our source code for this project freely available for download from the web pages of the SOMS chair. Furthermore, we assure that all source code is written by ourselves and is not violating any copyright restrictions.

\begin{center}

\bigskip


\bigskip


\begin{tabular}{@{}p{6cm}@{}p{6cm}@{}@{}p{6cm}@{}}
\begin{minipage}{6cm}
 \large Lucas Böttcher

\end{minipage}
& 
\begin{minipage}{6cm}
\large Simon Roth

\end{minipage}
&
\begin{minipage}{6cm}
\large Gabriela Schär

\end{minipage}
\end{tabular}
\end{center}
\newpage

%%%%%%%%%%%%%%%%%%%%%%%%%%%%%%%%%%%%%%%



% IMPORTANT
% you MUST include the ETH declaration of originality here; it is available for download on the course website or at http://www.ethz.ch/faculty/exams/plagiarism/index_EN; it can be printed as pdf and should be filled out in handwriting


%%%%%%%%%% Table of content %%%%%%%%%%%%%%%%%

\tableofcontents

\newpage

%%%%%%%%%%%%%%%%%%%%%%%%%%%%%%%%%%%%%%%



\section{Abstract}
\label{sec:abstract}

The process of continuous urbanization is noticeable in Switzerland like in the most other countries.\footnote[1]{see the autumn sunday lectures at ETH Science City (Die Stadt der Zukunft - die Zukunft der Stadt)} Besides the benefits of combining educational, medical, cultural and industrial centers in a small area, there are also some problems arising. The cities are growing, whereupon this growth in population causes an increasing use of the public transport system, which has to be adapted to the new conditions. The main goal of this project is the simulation of the Swiss rail network depending on population growth. The network is simulated with a biological inspired model based on Physarum polycephalum. This slime mold is a large single-celled amoeboid organism that forages for food sources. To maximize the searched area, it explores its environment with a relatively continuous foraging margin. It is forming different junctions and nodes to reduce the overall length of the connecting network \cite{network_tokyo}. This principle is adapted to the main railroads in Switzerland. 

\section{Individual contributions}
\begin{itemize}
  \item Lucas Böttcher: Implementation Shortest Connection, Report
  \item Simon Roth: Implementation Swiss Rail Network Simulation, Report
  \item Gabriela Schär: Processing of geographic data, Report
\end{itemize}


\section{Introduction and Motivations}
\label{sec:introduction}
Everyday thousands of commuters use the Swiss rail network to reach their workplace, school or university, mainly the same railroad is used. Seeing these facts we are going to ask, whether the actual situation of the network is the most efficient one. And related to that question, whether different connections between cities have maybe a higher performance.\\
\\
We assume, that the rail network is developed in a way that the connections between cities are the most efficient. Certain biological organisms connect their food sources in a similar way. In this project we simulate a biological organism creating its own efficient connections between the cities. This approach leads to the following main questions:

\begin{enumerate}
	\item Is the biological model a good approximation to simulate the network compared to reality?
	\item Are there any new built or destroyed connections in the network because of the future population growth?
\end{enumerate}

To answer these questions, the results of the simulations are compared with the Swiss rail network. For this comparison geographic data from geodata.ethz.ch~\cite{gis_data} are used. This data were processed with \textit{ArcGis 10.1} to get the basis for the simulation.

Cities were chosen based on the numbers of inhabitants (more than 10000) in 2011 and the presence of a railway station. Additional to these ones some other cities with less than 10000 inhabitants are added because of actual importance for the Swiss rail network\,\cite{bfs}. ~\\
~\\

\section{Description of the Model}
\label{sec:description}

The model is inspired by the Physarum polycephalum and the way this single-celled fungus searches for food. The organism have been subjected to successive rounds of evolutionary selection and have found an appropriate balance between efficiency, cost and resilience. The plasmodium contains a network of tubes, which enables chemical signals and nutrients to circulate through the organism. If some of the tubes have found food sources, this implies a positive feedback to the system. This tubes are getting thicker so the flux of nutrients increase. Other tubes which do not connect to food sources are shrinking and tend to disappear, because there is no flux available. Experiments showed two empirical rules. \textbf{1)} Tubes with no connection to a source disappear and \textbf{2)} if there are two tubes connecting the same source, the longer one disappears. This led to useful approaches to problem-solving like optimization of railroad systems or networks in general.~\\
~\\
The mathematical model is based on \textit{Physarum solver: A biologically inspired method of road-network navigation} \cite{network_model}. Figure \ref{fig:schema} shows the concept of the mathematical model. There are different nodes. The first two nodes corresponding to the food sources ($N_1$ and $N_2$) and other nodes ($N_3$, $N_4$ ... $N_j$). The junction between the node $N_i$ and $N_j$ is denoted as $M_{ij}$. The single nodes in the model are connected with circular tubes and their inner flux is measured(Figure \ref{fig:fluidd}). Therefore the fluid dynamics leads to the law of Hagen-Poiseuille, which allows to calculate the flux from the difference in pressure, the tube length and some constants. Starting from the fluid velocity in a cylindric tube \cite{kirb2010}:
\begin{figure}[H]
	\centering
	\includegraphics[width=10cm]{figures/figure2}
	\caption{Illustration of the Hagen-Poiseuille law for a circular tube, which connects the single nodes in our model.}
	\label{fig:fluidd}
\end{figure}

\begin{equation}
u(z)=-\frac{1}{4\eta}\frac{dp}{dz}(R^2-\rho^2)
\end{equation}
, where R is the radius of the cylinder and $\eta$ the viscosity. That expression can be integrated over a circle, and normalized to get a mean velocity:
\begin{equation}
\overline{u(z)}=-\frac{1}{4\pi\eta R^2}\frac{dp}{dz}\int_{0}^{2\pi}\int_{0}^{R}{(R^2-\rho^2)}\rho d\rho d\phi=-\frac{1}{8\eta}\frac{dp}{dz}R^2
\end{equation}
Assumeing that the pressure gradient is uniform and express the mean velocity by the flux divided by its cross section:
\begin{equation}
Q_{ij}=\frac{D_{ij}}{L_{ij}}\Delta p_{ij}
\end{equation}
, where $D_{ij}=-\frac{\pi R^4}{8\eta}$ is the conductance. So the Hagen-Poiseuille law is related to Ohms law, where a similar expression can be found.
\begin{figure}[H]
	\centering
	\includegraphics[width=6cm]{figures/figure1}
	\caption{\textbf{a)} Concept of the mathematical model with a source and sink node. The distance between the given nodes is a constant $d$. The centered node is not reachable.}
	\label{fig:schema}
\end{figure}

The variable $Q_{ij}$ stands for the flux through the junction $M_{ij}$ from $N_i$ to $N_j$. The flux $Q_{ij}$ is given by an approximately Poiseuille flow where $p_i$ is a pressure at the node $N_i$, $L_{ij}$ is the length and $D_{ij}$ is the conductivity of the junction $M_{ij}$ :

\begin{equation}
	\label{eq:1}
	Q_{ij}=\frac{D_{ij}}{L_{ij}}\left(p_i-p_j\right)
\end{equation}

By considering Kirchhoff's law at each node, there is:

\begin{equation}
	\label{eq:2}
	\sum_{i} Q_{ij}=0 \,\,\,\, \mathrm{if} \left(j\ne 1,2\right)
\end{equation}

$N_1$ is assumed to act as a source node and $N_2$ as a sink and $I_0$ represents the flux from the source node and is in this model constant. It follows:

\begin{equation}
	\label{eq:3}
	\sum_{i} Q_{i1}+I_0=0, \,\,\,\, \sum_{i} Q_{i2}-I_0=0
\end{equation}

To describe the thickness of the junctions we assume that the conductivity $D_{ij}$ changes in time according to the flux $Q_{ij}$:

\begin{equation}
	\label{eq:4}
	\frac{dD_{ij}}{dt}=f\left(\mid Q_{ij} \mid \right)-D_{ij}
\end{equation}

where $f\left(\mid Q \mid \right)$ is a increasing function with $f(0)=0$. Here $f\left(\mid Q \mid \right)$ is given by:

\begin{equation}
	\label{eq:5}
	f\left(\mid Q \mid \right)=\frac{\mid Q \mid^\gamma }{1+\mid Q \mid^\gamma}
\end{equation}

The network Poisson equation for the pressure is derived from the equations (\ref{eq:1}), (\ref{eq:2}) and (\ref{eq:3}) as followed:

\begin{equation}
	\label{eq:6}
	\sum_{i} \frac{D_{ij}}{L_{ij}}\left(p_i-p_j\right)= \begin{cases}
										-I_0 & \mathrm{for}\,\, j=1,\\
										I_0 & \mathrm{for} \,\,j=2,\\
										0 & \mathrm{otherwise}
										\end{cases}
\end{equation}

All variables $p_i$ can be determined by solving the equation system (\ref{eq:6}) when setting $p_2=0$ as a basic pressure level. With this assumption also each $Q_{ij}$ is obtained. They are defined by the variables $D_{ij}$ and $L_{ij}$  at each time step. Conductivity is closely related to the thickness of the junctions and so when a conductivity of a junction is zero, it disappears. The nodes $N_1$ and $N_2$ were randomly chosen at each time step.

\section{Implementation}
\subsection{ArcGis}
\label{sec:arcgis}

To get the basis matrix for the simulation, Swiss geographic data\,\cite{gis_data} are processed with \textit{ArcGis\,10.1}. Therefor the raster data were distinguished in different classes as described in Table \ref{tab:class}.

\begin{table}[H]
	\centering
	\caption{classes of data}
		\begin{tabular}{lll}
		\toprule
		Indentification & Class & Description \\
		\midrule
		0 & Ghostland & Cells in which the slime mold is not allowed to be\\
		& 		& (foreign coutries, lakes)\\
		1 & Free land & Cells in which the slime mold is allowed to grow\\
		2 & Junction & Cells in which the slime mold is located, \\
		& & processed in MATLAB\\
		3 & City & Cells which represents food sources\\
		\bottomrule
	\end{tabular}
\label{tab:class}
\end{table}

All the geographic data are given in vector format. So all lakes and the foreign countries can be erased. It would be possible to erase more types of covering (eg. rivers, rocks) where a train can not ride. The slope is in this case also neglected. So the mountains are not considered in the simulation. Then all the chosen cities are imported to \textit{ArcGis}. Therefor the coordinates (Y,X)\,\cite{coordinates} for each city were searched and a buffer of 2500m was laid on them. In this case no city will disappear when the vector data get converted to raster data with a cell size of 2500m. So after converting in raster data with the classes above, the basis matrix is exported as ASCII-File which can now used in MATLAB as simulation surface.
All the geographic data are in vector format. So all lakes and the foreign countries can be erased. It would be possible to erase more types of covering (eg. rivers, rocks) where a train can not ride. The slope is neglected in this case too. So the mountains are not considerated in the simulation. Then all the choosen cities are imported to \textit{ArcGis}. Therefor the coordinates (Y,X)\,\cite{coordinates} for each city are searched and a buffer of 2500m is layed on them. In this case no city will disappear when the vectordata get converted to rasterdata with a cellsize of 2500m. So after converting in rasterdata with the classes above, the basis matrix is exported as ASCII-File which can now used in MATLAB as simulation surface.

To compare the simulated results with the actual Swiss rail network, different types of railroads are neglected. Therefor every tunnel and every rail road which is not in use are not shown.



\begin{figure}[H]
	\centering
	\includegraphics[width=0.7\textwidth]{figures/map_2500_cities}
	\caption{Rastardata from ArcGis with cellsize of 2500m - white: Ghostland, green: Free land, brown: City}
	\label{fig:map_cities}
\end{figure}

\pagebreak
\subsection{MATLAB}

\subsubsection{Swiss Rail Network Simulation}

To simulate the Swiss rail network a cellular automata approach with a von Neumann neighborhood is used in this project because of the raster property of the model and the geographic data. This means, the cities are fixed nodes and all free land is populated with junctions. During the simulation junction cells with low conductivity are changed to free land. The basic structure of the network simulation is described in Figure \ref{fig:structure}.

\begin{figure}[H]
	\centering
	\includegraphics[width=0.7\textwidth]{figures/source_code}
	\caption{Structure of the network simulation implementation in MATLAB.}
	\label{fig:structure}
\end{figure}

The simulation can be separated in two tasks, first the computation of data and second the analysis of data. Because of the long duration of one simulation and the lack of a break condition, owning to quasi-stationary conditions, the two task are run separately.

The \textbf{main.m} script holds all necessary informations like parameters, constants, folder and file paths to run the simulation. A backup function is implemented, which can be turned on an off, because of program crashes during testing. The three input files geodata.txt, population.txt and growth.txt, which have to be in an appropriate form without header row, are used to run the simulation. All input data are processed to input variables for the main loop in the \textbf{initial.m} function. This function initializes the folder structure of the simulation, converts the input data, restores the backup if turned on and passes all the data to the main loop. The main tasks of the \textbf{loop.m} function are to compute the linear equation system (equation \ref{eq:6}) and to solve the equations. For the equation system a index for every node and junction cell has to be computed. Due to performance loss by using the built in sub2ind MATLAB function the \textbf{varnum.m} function is implemented, which returns the index from a cell depending on the coordinates. After solving the linear equation system the new conductivities and an updated path can be computed. Because of the machine precision (eps) a junction cell already disappears before the conductivity is zero. This is implemented after noticing this behavior in program tests. The simulation of the Swiss rail network needs a lot of time depending on the size of the input data, so a output file is implemented to control the progress and already computed results. To make a decision about the ending of the simulation a visual break conditions is implemented to see how far the network is. This picture is saved by the \textbf{draw.m} function and displays the actual network.

The \textbf{main\_analysis.m} script contains all the variables to compute the final simulation results, which are computed in the \textbf{analysis.m} function. The final network and overall conductivity is converted to ASCII files, which are used to compare the different scenarios in \textit{ArcGIS}. To analyze the development of the path length over time these two variables are stored in vectors to be plotted afterwards.

The whole program is attached to the appendix\,\ref{sec:network_simulation}.



\subsubsection{Shortest Connection}
\label{sec:shortest_connection}
For evaluating the path length given by the Physarum simulation data, a program is used, which computes the shortest path length between two given points on the map. A standard algorithm \cite{gaertner2010} for implementing the function in MATLAB is used. In general the code uses a start coordinate from an input file, related to the map matrix A with the defined neighborhood, i.e.:

\begin{lstlisting}
function [B,short_length] = shortest_path(A,points,neigh)
\end{lstlisting}

\null

By calling the function with the necessary arguments a matrix B is generated, which contains the given map, without cities or earlier computed ways, but with the drawn shortest path. The shortest path length \textbf{int} is a second output argument of the function. The algorithm explained in a nutshell can be described with the following words: After cleaning the computed matrix from cities and ways, the start entry is coded with two. As a result the matrix contains only zeros (ghostland), ones (free land) and the start entry two. The algorithm iterates over the whole matrix entries and continues in the loop. If the entry is not one, the algorithm jumps to the next cell:

\begin{lstlisting}
if B(k,l) ~= 1
	continue;
end
\end{lstlisting}

\null

The cells are numerated starting by two and the next reachable cell gets the next natural number. This iterates with the given neighborhood:

\begin{lstlisting}
if B(m,n) == step+1
	B(k,l) = step+2;
	iter = 1;
end
\end{lstlisting}

\null

The variable \textbf{iter} shows if there is any neighbor (=1), if not the loops stops:

\begin{lstlisting}
if ~iter
	break;
end
\end{lstlisting}

\null

At the end the output is a numerated path, which can be used to calculate the track length or to use a backtracking algorithm for getting the path in the output matrix B. The whole program is attached to the appendix\,\ref{sec:shortest_path}.


\pagebreak
\section{Simulation Results and Discussion}
\label{sec:results}
All the simulations are ran with the same parameters and with a cell size of 2500m.

In Figure\,\ref{fig:path} and Figure\,\ref{fig:conductivity} one can see the results of the first MATLAB simulation and the actual rail network. In this simulation the assumption made, is that every city has an equal number of inhabitants. The chosen path is shown in Figure\,\ref{fig:path}. Under consideration of the chosen cities and the neglected slope this simulated path is a good approximation of the real rail network. The plot in Figure \,\ref{fig:plottrail} shows the development of the overall path length over time. At first the function is on a constant level and then the length decreases in an exponential way. With more time steps,  the function has to iterate over less cells and so less time steps are needed to proceed. Around time step 40 the simulation tries to get in a steady state but small perturbations prevents it reaching an equilibrium.

In Figure\,\ref{fig:conductivity} is shown the conductivity which goes along with this simulated path. It describes which line is used the most (e.i. Bern-Solothurn-Sursee-Lucern-Zug-Zurich).

For verifying the result this first simulation rans twice, each time with the same result, what can be expected from the deterministic algorithm.



\begin{figure}[H]
	\centering
	\includegraphics[width=0.7\textwidth]{figures/path_railway}
	\caption{Simulated path of Physarum polycephalum and the actual Swiss rail network}
	\label{fig:path}
\end{figure}

\begin{figure}[H]
	\centering
	\includegraphics[width=0.5\textwidth]{figures/plottrail1}
	\caption{Development of the overall path length according to Figure\,\ref{fig:path}; path length in normalized numbers (to the value of convergence) of cells containing slime mold}
	\label{fig:plottrail}
\end{figure}

\begin{figure}[H]
	\centering
	\includegraphics[width=0.7\textwidth]{figures/conductivity_railway}
	\caption{Simulated conductivity of Physarum polycephalum and the actual Swiss rail network}
	\label{fig:conductivity}
\end{figure}
\pagebreak
To answer the second main question, a second simulation was started. This time, each city has its own number of inhabitants corresponding to the assumed population in 2050 calculated with a linear growth\,\cite{bfs}. The result is shown in Figure\,\ref{fig:path_2050}. Comparing to the actual path there are just slight differences. Except of the junction between St.\,Moritz-Bellinzona and Bern-Solothurn, which disappears in the simulation of 2050. This could be explained with the low population of both of these cities (St.\,Moritz/Bellinzona) and in this case the connection gets less important. The disappeared connection between Bern and Solothurn can be explained with different numbers of time steps over which the simulation iterates (first simulation: 5973 time steps, second simulation: 7318 time steps). With more time steps the junction would also disappear in the first simulation.
To answer the second main question, a second simulation is ran. This time, each city has its own number of inhabitants corresponding to the assumend population in 2050 calculated with a linear growth\,\cite{bfs}. The result is shown in Figure\,\ref{fig:path_2050}. Comparing to the actual path there are just slight differences. Except of the junction between St.\,Moritz-Bellinzona and Bern-Solothurn, which disappear in the simulation of 2050. This could be explained with the low population of both of these cities (St.\,Moritz/Bellinzona) and in this case the connection gets less important. The disappeared connection between Bern and Solothurn can be explained with different numbers of time steps over which the simulation itterates (first simulation: 5973 time steps, second simulation: 7318 time steps). With more time steps the junction would also disappear in the first simulation. The developing of the overall path length is in both simulation equal. The different length at the beginning is caused of the normalization to the end length.

Comparing the plots of the conductivity there are just slight differences too. In general, the conductivity is lower than in the first simulation. That depends on lower initial flux caused by the different populations, which is higher in the first simulation. The line with the highest conductivity is longer in the simulation of 2050 than in the first simulation as shown in Figure\,\ref{fig:conductivity_2050}. The most frequently used line goes now from Lausanne over Bern, Solothurn, Lucern, Zug to Zurich. This leads to the conclusion that the Swiss rail network is not going to change in the future.

\begin{figure}[H]
	\centering
	\includegraphics[width=0.6\textwidth]{figures/path_railway_2050}
	\caption{Simulated path in 2050 of Physarum polycephalum and the actual Swiss rail network}
	\label{fig:path_2050}
\end{figure}

\begin{figure}[H]
	\centering
	\includegraphics[width=0.5\textwidth]{figures/plottrail2}
	\caption{Development of the overall path length according to Figure\,\ref{fig:path_2050}; path length in normalized numbers (to the value of convergence) of cells containing slime mold}
	\label{fig:plottrail_2050}
\end{figure}

\begin{figure}[H]
	\centering
	\includegraphics[width=0.7\textwidth]{figures/conductivity_railway_2050}
	\caption{Simulated conductivity in 2050 of Physarum polycephalum and the actual Swiss rail network}
	\label{fig:conductivity_2050}
\end{figure}

The question comes up, if the simulated path of the slime mold is the shortest connection between two cities. To verify this assumption a second program is used (compare \ref{sec:shortest_connection}). The result is that the lengths between the cities are in both simulations equal as shown in Figure\,\ref{fig:short}. This make sense, because the shortest path is often the one with the highest efficiency.

\begin{figure}[H]
	\centering
	\includegraphics[width=12cm]{figures/figure3}
	\caption{\textbf{a)} Simulated path length from the Physarum simulation code. \textbf{b)} Zooming into the red square is on the left hand side the Physarum simulation path length, compared to the right hand side computed shortest path length expample.}
	\label{fig:short}
\end{figure}
\label{sec:matlab}


\section{Summary and Outlook}
\label{sec:summary}

The main goal of this project was to simulate the Swiss rail network based on a biological inspired model. In a first simulation the path of the Swiss rail network is simulated and in a second simulation the influence on the network based on the population in the different cities is investigated. So both main questions can be answered.

\begin{enumerate}
\item The simulation leads to a good approximation of the real Swiss rail network under consideration that different surface covers and the slope is neglected.
\item Depending on the population growth there are built no new connections but one is disappeared.
\end{enumerate}

The model based on Physarum polycephalum leads in general to a good approximation of a rail network. But the utilization of a rail network does not only depend on the number of people living in a city. In this project, this is the only parameter which the model uses to decide which connection is efficient. There are also other impacts which make a specific connection more efficient or more attractive to use. For example how fast can a train ride and following this how long lasts a journey. Especially in the Alps the number of tourists and the touristic villages with less inhabitants are an important factor.\\

There are different points to continue with this project or maybe to enlarge it in the future. With more time and hardware capacity it is possible to do a lot more simulations to find the best set of parameters. Another interesting point would be to simulate the rail network with more restrictions like the slope or other surface coverings. It is also possible to simulate different population growth scenarios to see the consequences in the resulting rail network. Another way to use the model is to preset the actual rail network and to simulate the utilization of the different rail lines. Because each city has another impact on the whole network (e.g. considering international connections like air ports ore border crossings) one can define weighing factors, which respect more parameters, than the population.


\newpage
\appendix
\section{Matlab-Code}

\subsection{Network Simulation}
\label{sec:network_simulation}

\subsubsection{main.m}

\lstinputlisting{../../code/network_simulation/main.m}

\null
\null

\subsubsection{initial.m}

\lstinputlisting{../../code/network_simulation/initial.m}

\null
\null

\subsubsection{loop.m}

\lstinputlisting{../../code/network_simulation/loop.m}

\null
\null

\subsubsection{varnum.m}

\lstinputlisting{../../code/network_simulation/varnum.m}

\null
\null

\subsubsection{draw.m}

\lstinputlisting{../../code/network_simulation/draw.m}

\null
\null

\newpage
\subsubsection{main\_analysis.m}

\lstinputlisting{../../code/network_simulation/main_analysis.m}

\null
\null

\subsubsection{analysis.m}

\lstinputlisting{../../code/network_simulation/analysis.m}

\null
\null

\subsection{Shortest Path}
\label{sec:shortest_path}

\lstinputlisting{../../code/short_path/shortest_path.m}

\subsection{List of Cities}
\begin{figure}[H]
	\centering
	\includegraphics[width=1.0\textwidth]{figures/city_list1}
\end{figure}

\begin{figure}[H]
	\centering
	\includegraphics[width=1.0\textwidth]{figures/city_list2}
\end{figure}

\begin{figure}[H]
	\centering
	\includegraphics[width=1.0\textwidth]{figures/city_list3}
\end{figure}

\begin{figure}[H]
	\centering
	\includegraphics[width=1.0\textwidth]{figures/city_list4}
\end{figure}

\begin{figure}[H]
	\centering
	\includegraphics[width=1.0\textwidth]{figures/city_list5}
\end{figure}




\includepdf{figures/declaration.pdf}


\bibliographystyle{plain}
\bibliography{matlabbib}






\end{document}  



 
